\section{Principal Queries}
\textcolor{red}{[Write some of the queries to be performed to satisfy your functional requirements. 3-4 queries are enough, try to use the techniques seen at lecture (aggregate functions, group by, subqueries,…)]}

\begin{lstlisting}[language=SQL,
keywordstyle=\color{blue},
stringstyle=\color{mauve},
showstringspaces=false,
basicstyle=\ttfamily\footnotesize]
SELECT t1.attr1
FROM table1 as t1
    LEFT JOIN table2 as t2 ON t1.attr2=t2.attr1 
WHERE t1.attr3=1


List all contracts between a specific manager and a specific supplier.

SELECT c.contract_id,
       c.description,
       c.contract_date AS contract_date,
       c.expiration_date AS expiration_date,
       e.first_name AS manager_name,
       e.last_name AS manager_surname
		FROM tagms.contract AS c
		    INNER JOIN tagms.employee AS e ON c.employee_id = e.tax_number
		    INNER JOIN tagms.supplier AS s ON c.supplier_id = s.vat_number
WHERE e.tax_number='FGDVSF30C62D012T'
  AND s.supplier_name='Reg s.r.l.';



Show all the details of the employees in the organization.

SELECT e.tax_number,
       e.first_name,
       e.last_name,
       e.phone_number,
       e.email_address,
       e.birth_date,
       e.hiring_date,
       r.name,
       d.name
FROM tagms.employee AS e
    INNER JOIN tagms.work AS w ON e.tax_number = w.employee_id
    INNER JOIN tagms.department AS d ON w.department_id = d.department_id
    INNER JOIN tagms.role AS r ON r.role_id = e.role_id
WHERE e.still_working = TRUE;



After inserting a new lot with identifier Lot_id (see the "populate" section), decrease the quantity of the items involved in the production of a lot.

UPDATE tagms.item AS i SET
    quantity = c.quantity
FROM (
        SELECT i.item_id, i.quantity - l.product_quantity * m1.quantity AS quantity FROM tagms.lot AS l
            INNER JOIN tagms.made_up_of_1 AS m1 ON l.product_id = m1.product_id
            INNER JOIN tagms.item AS i ON m1.item_id = i.item_id
        WHERE l.lot_id = '3'
    UNION
        SELECT i.item_id, i.quantity - l.package_quantity * m2.quantity AS quantity FROM tagms.lot AS l
            INNER JOIN tagms.made_up_of_2 AS m2 ON l.package_id = m2.package_id
            INNER JOIN tagms.item AS i ON m2.item_id = i.item_id
        WHERE l.lot_id = '3'
     )
    AS c(item_id, quantity)
WHERE c.item_id = i.item_id
RETURNING i.item_id, name, i.quantity, minimum_quantity;



After inserting a new contract with identifier Contract_id, the quantities of items in stock must be incremented in the delivery date.

UPDATE tagms.item AS i SET
    quantity = c.quantity
FROM (
    SELECT i.item_id, i.quantity + s.purchased_quantity AS quantity FROM tagms.contract AS c
        INNER JOIN tagms.specify AS s ON c.contract_id = s.contract_id
        INNER JOIN tagms.item AS i ON s.item_id = i.item_id
    WHERE c.contract_id = '4'
     )
     AS c(item_id, quantity)
WHERE c.item_id = i.item_id
RETURNING i.item_id, name, description, i.quantity, minimum_quantity, item_category_id;



Lists all available lots (unsold and that won't expire in 6 months) containing a particular product having a given Product\_id as identifier, sorted by expiration date (oldest lots must be sold first).

SELECT l.lot_id,
       l.expiration_date AS expiration_date,
       l.product_id,
       l.product_quantity,
       ROUND(l.lot_price * (1 + l.vat/100) * (1 - l.lot_discount/100), 2) AS gross_price
FROM tagms.lot AS l
    LEFT OUTER JOIN tagms.draws_from AS df ON l.lot_id = df.lot_id
WHERE l.product_id = '6'
  AND l.expiration_date > (current_date + interval '6 months')
  AND df.order_id IS NULL
ORDER BY l.expiration_date ASC;



Get the net sales and paid taxes in a given time interval.

SELECT SUM(o.net\_price) AS net\_sales, SUM(o.taxes) AS taxes FROM tagms.order AS o
    WHERE DATE(o.order\_date) >= '2021-01-01' AND
          DATE(o.order\_date) <= '2021-12-31' AND
          o.order\_paid = TRUE;



Get the cost of materials in a given time interval

SELECT SUM(sp.price) AS cost_of_material FROM tagms.specify as sp
    INNER JOIN tagms.contract AS c ON c.contract_id = sp.contract_id
WHERE c.contract_date >= '2021-01-01' AND c.contract_date <= '2021-12-31';




Get the production cost in a given time interval

SELECT SUM(p.production_cost * l.product_quantity) AS production_cost FROM tagms.lot AS l
    JOIN tagms.product AS p ON l.product_id = p.product_id
    JOIN tagms.draws_from AS df ON l.lot_id = df.lot_id
    JOIN tagms.order AS o ON df.order_id = o.order_id
WHERE DATE(o.order_date) >= '2021-01-01' AND DATE(o.order_date) <= '2021-12-31';



When inserting an order, the operator can choose a custom price (e.g., decided with the customer)
or use the following query.
Given an order with Order_id, compute the order's net_price and total taxes.
Note: the given Order_id must be written twice in the query

UPDATE tagms.order AS o
SET net_price = tmp.net_price,
    taxes = tmp.taxes
FROM (
         SELECT SUM(l.lot_price * (1 - l.lot_discount / 100)) AS net_price,
                SUM(l.lot_price * l.VAT/100) as taxes
         FROM tagms.draws_from AS df
                  INNER JOIN tagms.lot AS l ON df.lot_id = l.lot_id
        WHERE df.order_id = '2'
    ) AS tmp(net_price, taxes)
WHERE o.order_id = '2'
RETURNING o.order_id, o.net_price, o.taxes;

Note: the given Order\_id must be written twice in the query


Given an item (which is used to create a product) having Item_id as identifier and a time interval (actually, two dates),
find the total quantity of that item that has been used for production or packaging during that time.

SELECT SUM(l.product_quantity * m1.quantity) AS quantity FROM tagms.made_up_of_1 AS m1
    INNER JOIN tagms.lot AS l ON m1.product_id = l.product_id
    INNER JOIN tagms.draws_from AS df ON l.lot_id = df.lot_id
    INNER JOIN tagms.order AS o ON df.order_id = o.order_id
WHERE m1.item_id = '1'
  AND DATE(o.order_date) >= '2021-01-01'
  AND DATE(o.order_date) <= '2021-12-31';



Given an item (which is used to create a product) having Item\_id as identifier and a time interval (actually, two dates), find the total quantity of that item that has been used for production or packaging during that time.

SELECT SUM(l.product\_quantity * m1.quantity) AS quantity FROM tagms.made\_up\_of\_1 AS m1
    INNER JOIN tagms.lot AS l ON m1.product\_id = l.product\_id
    INNER JOIN tagms.draws\_from AS df ON l.lot\_id = df.lot\_id
    INNER JOIN tagms.order AS o ON df.order\_id = o.order\_id
WHERE m1.item\_id = '9'
  AND DATE(o.order\_date) >= '2021-01-01'
  AND DATE(o.order\_date) <= '2021-12-31';



Return the list of items which have stock quantity lower than the minimum one.

SELECT * FROM tagms.item AS i
WHERE i.quantity < i.minimum\_quantity;




Return the list of unsellable lots which are in stock (that will expire in less than 6 months)

SELECT
       l.lot_id,
       l.expiration_date AS expiration_date,
       l.product_id,
       l.product_quantity,
       l.package_id,
       l.package_quantity,
       l.lot_discount,
       l.vat,
       l.lot_price
       FROM tagms.lot AS l
    LEFT OUTER JOIN tagms.draws_from AS df ON l.lot_id = df.lot_id
WHERE l.expiration_date <= (current_date + interval '6 months')
  AND df.order_id IS NULL;

\end{lstlisting}

