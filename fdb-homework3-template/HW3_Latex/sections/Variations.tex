\section{Variations to the Relational Schema}
%\textcolor{red}{[Describe here variations and/or corrections to the relational schema of the previous homework, if present. Otherwise, report only the relational schema.]}

We have found some typos within the Data Dictionary section in the previous homework. Below are listed the introduced corrections:
\begin{itemize}
	\item In Employee the ``Registration\_date\_ID'' attribute has been renamed to ``Registration\_date''.
	\item In Draws\_from, the description of the ``Lot\_ID" attribute is changed from ``Foreign Key to Lot, Primary key to Lot'' to ``Foreign Key to Lot, Primary key''
	\item In Ship, the ``Employee\_ID'' attribute is not a serial but a text type. In particular, a new domain called ``taxnumber'' has been defined, to which the Employee entity ID is associated.
\end{itemize}

\noindent Hereafter are reported several clarifications of some choices made during the creation of the DB. The choice to use the data type ``numeric'' rather than ``money'' in the price domain is dictated by the company's need to use a more precise decimal notation. In fact, micro price variations may be necessary, which would not be allowed with the ``money'' type. The company is therefore able to specify a price up to 3 decimal digits. Furthermore, since the company operates only in Italy, the currency used within the system to store the various prices and costs is the euro. Using money, on the other hand, several conversions would have been necessary, as the default currency is in dollars. Furthermore, some constraints (e.g., NOT NULL), have been included within some domain definitions. For example, the ``price'' domain, which is used in all the fields where an amount must be specified, has been set as ``NOT NULL'' in the ``CREATE DOMAINS'' section. It is possible to verify from the ``CREATE TABLE'' section in the ``Physical.tex'' file how the choice of unique identifiers has been applied as much as possible (e.g., the VAT number is used for the customer, while the TAX number is used for the employee). In any other case, a self-increasing serial integer attribute has been used.