\subsection{Transformation of the Entity-Relationship Schema}

\subsubsection{Redundancy Analysis}

The Employee and Order entities, related to each other through the relationships Place and Ship, do not form a cycle because:
- Only the Employee with the Salesman role can place the order;
- Only the Employee with the Worker role can ship the order.
- Eliminating the Place or Ship relationship implies a violation of functional requirements and a loss of information (e.g., who is the seller who places the order or who is the worker who ships it).

The Lot, Product, Package, and Item entities do not form a loop because:
- Eliminating a "Made up of" relationship implies a violation of functional requirements and a loss of information (eg, from which items, and in what quantity, a product or package is composed);
- Eliminating the "Stocked" relationship implies a violation of functional requirements and a loss of information (e.g., it is not possible to know which products and packages are contained in a lot)

The ER schema presents the derivate attributes:
- "Quantity" (of Item): used to keep in memory the current quantity of items.
- "Lot_price" (of Lot): used to keep in memory the price of the lot at the moment of preparation.
- "Net_price" (of Order): used to keep in memory the total net price of an order.
- "Taxes" (of Order): used to keep in memory the total taxes of an order.

We report below the analysis of the database load to check whether keeping these attributes or not.

\subsubsection{Choice of Principal Identifiers}