\subsection{Transformation of the Entity-Relationship Schema}
\subsubsection{Redundancy Analysis}
The Employee and Order entities, related to each other through the relationships Place and Ship, do not form a cycle because:
\begin{itemize}
        \item Only the Employee with the Salesman role can place the order;
        \item Only the Employee with the Worker role can ship the order;
        \item Eliminating the Place or Ship relationship implies a violation of functional requirements and a loss of information (e.g., who is the seller who places the order or who is the worker who ships it).
\end{itemize}
The Lot, Product, Package, and Item entities do not form a loop because:
\begin{itemize}
        \item Eliminating a "Made up of" relationship implies a violation of functional requirements and a loss of information (eg, from which items, and in what quantity, a product or package is composed);
        \item Eliminating the "Stocked" relationship implies a violation of functional requirements and a loss of information (e.g., it is not possible to know which products and packages are contained in a lot).
\end{itemize}
The ER schema presents the derivate attributes:
\begin{itemize}
        \item "Quantity" (of Item): used to keep in memory the current quantity of items;
        \item "Lot\_price" (of Lot): used to keep in memory the price of the lot at the moment of preparation;
        \item "Net\_price" (of Order): used to keep in memory the total net price of an order;
        \item "Taxes" (of Order): used to keep in memory the total taxes of an order.
\end{itemize}
\vspace{0.25em}
We report below the analysis of the database load to check whether keeping these attributes or not.
\subsubsection{Choice of Principal Identifiers}
The schema does not contain external identification cycles and the main identifiers comply with the selection criteria.