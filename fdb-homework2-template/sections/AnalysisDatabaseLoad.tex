\subsection{Analysis of Database Load}
The load analysis is divided in two parts: the first, to decide whether to store ``lot\_price'' into the ``Lot'' entity, or computing it when necessary via the relationship ``Stocked''; the second, to decide whether to store ``quantity'' into the ``Made up of (2)'' relationship, or computing it when necessary via the relationship ``Made up of (2)''. The second part of the analysis is done also for the other relationship ``Made up of (1)'' and leads to the same results.

\begin{center}
	\begin{tabular}[!h]{ | c | c | c | c | }
		\hline
		\textbf{Operation} & \textbf{Description} & \textbf{Frequency} & \textbf{Type} \\ \hline
		$ \textrm{O}_\textrm{1} $ : Insert new lot & Store data about a newly packaged lot. & 25/week & Online \\ \hline
		$ \textrm{O}_\textrm{2} $ : Compute order price  & Compute order price from lot price & 25/week & Online \\\hline
		$ \textrm{O}_\textrm{3} $ : Create new package & Create new package from items & 100/week & Batch \\\hline
		$ \textrm{O}_\textrm{4} $ : Compute the quantity & Compute the item quantity needed for creating the package & 7/week & Batch \\\hline
	\end{tabular}
\end{center}
\newpage
\begin{table}[!h]\caption{O1 Without redundancy}
	\begin{center}
		\begin{tabular}{| c | c | c | c | c |}
			\hline
			\textbf{Concept} & \textbf{Construct} & \textbf{Access} & \textbf{Type} & \textbf{Average Access} \\ \hline
			Product & Entity & 1 & R & 1 x 25 x 1 = 25 \\ \hline
			Lot & Entity & 1 & W & 1 x 25 x 2 = 25 \\ \hline
			Stocked & Relationship & 1 & W & 1 x 25 x 2 = 25 \\ \hline
			\multicolumn{3}{|c|}{\textbf{Total Access}} & \multicolumn{2}{|c|}{\textbf{75}} \\ \hline
		\end{tabular}
	\end{center}
\end{table}
\begin{table}[!h]\caption{O1 With redundancy}
	\begin{center}
		\begin{tabular}{| c | c | c | c | c |}
			\hline
			Product & Entity & 1 & R & 1 x 25 x 1 = 25 \\ \hline
			Lot & Entity & 1 & W & 1 x 25 x 2 = 25 \\ \hline
			Stocked & Relationship & 1 & W & 1 x 25 x 2 = 25 \\ \hline
			\multicolumn{3}{|c|}{\textbf{Total Access}} & \multicolumn{2}{|c|}{\textbf{75}} \\ \hline
		\end{tabular}
	\end{center}
\end{table}
\begin{table}[!h]\caption{O2 Without redundancy}
	\begin{center}
		\begin{tabular}{| c | c | c | c | c |}
			\hline
			\textbf{Concept} & \textbf{Construct} & \textbf{Access} & \textbf{Type} & \textbf{Average Access} \\ \hline
			Product & Entity & 1 & R & 1 x 25 x 1 = 25 \\ \hline
			Lot & Entity & 1 & W & 1 x 25 x 2 = 25 \\ \hline
			Stocked & Relationship & 1 & W & 1 x 25 x 2 = 25 \\ \hline
			\multicolumn{3}{|c|}{\textbf{Total Access}} & \multicolumn{2}{|c|}{\textbf{75}} \\ \hline
		\end{tabular}
	\end{center}
\end{table}
\begin{table}[!h]\caption{O2 With redundancy}
	\begin{center}
		\begin{tabular}{| c | c | c | c | c |}
			\hline
			\textbf{Concept} & \textbf{Construct} & \textbf{Access} & \textbf{Type} & \textbf{Average Access} \\ \hline
			Lot & Entity & 1 & W & 1 x 25 x 2 = 25 \\ \hline
			\multicolumn{3}{|c|}{\textbf{Total Access}} & \multicolumn{2}{|c|}{\textbf{25}} \\ \hline
		\end{tabular}
	\end{center}
\end{table}
We can see that $ \textrm{O}_\textrm{1} $, with or without redundancy, necessitate the same amount of operations, while from $ \textrm{O}_\textrm{2} $ we can assess that the number of operations without redundancy is tripled with respect to the case with redundancy. Hence, the attribute ``lot\_price'' of the entity ``Lot'' should to be kept.
\begin{table}[!h]\caption{Without redundancy O3}
	\begin{center}
		\begin{tabular}{| c | c | c | c | c |}
			\hline
			\textbf{Concept} & \textbf{Construct} & \textbf{Access} & \textbf{Type} & \textbf{Average Access} \\ \hline
			Package & Entity & 1 & W & 1 x 100 x 2 = 200 \\ \hline
			Made up of (2) & Relationship & 1 & W & 1 x 100 x 2 = 200 \\ \hline		Item & Entity & 1 & R & 1 x 100 x 1 = 100 \\ \hline		
			\multicolumn{3}{|c|}{\textbf{Total Access}} & \multicolumn{2}{|c|}{\textbf{300}} \\ \hline
		\end{tabular}
	\end{center}
\end{table}
\newpage
\begin{table}[!h]\caption{With redundancy O3}
	\begin{center}
		\begin{tabular}{| c | c | c | c | c |}
			\hline
			\textbf{Concept} & \textbf{Construct} & \textbf{Access} & \textbf{Type} & \textbf{Average Access} \\ \hline
			Package & Entity & 1 & W & 1 x 100 x 2 = 200 \\ \hline
			Made up of (2) & Relationship & 1 & W & 1 x 100 x 2 = 200 \\ \hline		Item & Entity & 1 & R & 1 x 100 x 1 = 100 \\ \hline		
			\multicolumn{3}{|c|}{\textbf{Total Access}} & \multicolumn{2}{|c|}{\textbf{300}} \\ \hline
		\end{tabular}
	\end{center}
\end{table}
\begin{table}[!h]\caption{Without redundancy O4}
	\begin{center}
		\begin{tabular}{| c | c | c | c | c |}
			\hline
			\textbf{Concept} & \textbf{Construct} & \textbf{Access} & \textbf{Type} & \textbf{Average Access} \\ \hline
			Package & Entity & 1 & R & 1 x 7 x 1 = 7 \\ \hline
			Made up of (2) & Relationship & 1 & R & 1 x 7 x 1 = 7 \\ \hline		Item & Entity & 1 & R & 1 x 7 x 1 = 7 \\ \hline		
			\multicolumn{3}{|c|}{\textbf{Total Access}} & \multicolumn{2}{|c|}{\textbf{21}} \\ \hline
		\end{tabular}
	\end{center}
\end{table}
\begin{table}[!h]\caption{With redundancy O4}
	\begin{center}
		\begin{tabular}{| c | c | c | c | c |}
			\hline
			\textbf{Concept} & \textbf{Construct} & \textbf{Access} & \textbf{Type} & \textbf{Average Access} \\ \hline
			Made up of (2) & Relationship & 1 & R & 1 x 7 x 1 = 7 \\ \hline	
			\multicolumn{3}{|c|}{\textbf{Total Access}} & \multicolumn{2}{|c|}{\textbf{7}} \\ \hline
		\end{tabular}
	\end{center}
\end{table}
We can see that $ \textrm{O}_\textrm{3} $, with or without redundancy, necessitate the same amount of operations, while from $ \textrm{O}_\textrm{4} $ we can assess that the number of operations without redundancy is tripled with respect to the case with redundancy, as well. Hence, the attribute ``quantity'' of the relationship ``made up of (2)'' should to be kept.