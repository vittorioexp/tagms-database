\subsection{External Constraints}
\begin{itemize}
\item The company decides the total discount to apply to a specific lot. This discount expresses a percentage and is a number between 0 and 100. Furthermore, the company is able to take into account changes in VAT.
\item The company decides the value of the price increase according to company policies. This increase ("Price\_increase") must take on a value greater than or equal to 1. The price of a product is calculated as a multiplication between the cost of production and the price increase.
\item The price of a specific lot ("Lot\_price") is calculated as the multiplication of the price of the product (stored in the lot) multiplied by its quantity. It is important to underline that the "Lot\_price" attribute cannot be derived through an online operation as the fields on which it depends ("Production\_Cost" and "Price\_increase") can be modified over time and therefore give discordant results at different times.
\item The total net amount of the order ("Net\_price") must be calculated as the sum, for each lot $i$ included, of the $ \mathrm{Lot\_price}_i $ * (1 - $ \mathrm{Lot\_discount}_i $ / 100). The total taxes are calculated as the sum, for each lot $i$, of the $ \mathrm{Lot\_price}_i $ * $ \mathrm{VAT}_i $ / 100.
\item The "Quantity" attribute of the "Item" entity is derived. When a new contract is signed, in the delivery date (specified by the attribute "Delivery\_date" of "Contract") the quantity of each item in the contract is incremented accordingly based on the "Purchased\_Quantity" attribute in the relationship "Specify". When a new lot is inserted in the system, each quantity of each item, that constitute the product in the lot, is decremented by a number equal to the multiplications between "Product\_quantity" in "Stocked" and "Quantity" in "Made\_up\_of(1)". The same applies to "Package" entity where the decreasing factor is calculated from the product between "Package\_quantity" and "Quantity" in "Made\_up\_of (2). Immediately after this, for each item involved, the "Quantity" attribute is compared with the "Minimum\_quantity" attribute so a manager will get notified to avoid shortages. Furthermore, the worker, before processing the order, verifies that there are sufficient quantities of items to satisfy the request.
\item The units of measure used by the company are specified in detail in the entity table.
\item Only employees that are sill working (i.e. those who have the "Still\_working" attribute set to "True") in the company can access the system.
\item Sellers can keep track of the order status on their own, also updating the payment status once the bill is settled. By default, the boolean "Order\_paid" is false. 
\item Only the Manager can insert new contracts, so the Employee who takes part in the "stipulate" relationship must have the role equal to "Manager". 
\item Only the Salesman can insert new orders, so the Employee who takes part in the "place" relationship must have the role equal to "Salesman". 
\item Only the Worker can ship orders, so the Employee who takes part in the "ship" relationship must have the role equal to "Worker".
\item A product must consist of one or more items having a certain Item\_Category (e.g., "ingredient" or "glass bottle"). A package must consist of one or more items having a certain Item\_Category (e.g., "box", "plastic tape" or "polystyrene"). 
\item The quantities of products contained in a lot belong to a finite set (e.g. 25, 50, 100), so the customer cannot order a lot with an arbitrary quantity of products.
\item An expired lot cannot be sold.
\end{itemize}
