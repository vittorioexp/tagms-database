\subsection{Functional Requirements Satisfaction Check}

The DBMS has to be able to:
\begin{itemize}
	\item \textbf{store all the details of the employees, customers and suppliers in the organization:} Entities Employee and Role store data related to the employees. Entity Customer has details about the customers and entity Supplier has data related to the Supplier.
	\item \textbf{allow the employees to update their personal information:} Entity Employee has some attributes as Email\_address, Password or Phone\_number which can be changed. Employees can access the system using their credentials which are Email\_address and Password and change this data.
	\item \textbf{store details of all on-hand products in the warehouse such as item code, item description, quantity and expiration date:} Attributes ID\_Product, Description, Expiration\_Date from entity Product and ID\_Item, Description from entity Item show this data. Secondly the amount of each product is shown in attribute Element\_quantity of entity Inventory.  
	\item \textbf{allow the employees to log into the system and enter the inbound items they received with information item code, item description, quantity, expiration date and supplier:} With attributes Email\_address and Password employees log in the application and insert this data in Entities Item and Inventory. \textcolor{Red}{Review}
	\item \textbf{show and generate the list of inbound and outbound transactions:} the inbound transactions can be derived from instances of the Contract entity, the outbound transactions can be derived from instances of the Order entity.
	\item \textbf{allow the employees to log into the system and enter the outbound transaction needed for the issuance of the products in the production and shipment to the customers:} Salesmen are responsible of entering outbound transactions as instances of the Order entity with proper attributes values, specifically Product\_List, Quantities\_List (has at index \textit{i} the quantity of purchased items for product at index \textit{i} of Product\_List), Address and ID\_Order which identifies the instance.
\item \textbf{inventory stocks will be automatically updated whenever there are inbound and outbound transactions:} the update is executed automatically when an inbound transaction happens by inspecting each ID\_Product \textbf{x} and associated Quantity \(Q_x\) in the relative Contract, then for each said \textbf{x} its Quantity attribute in the entity Inventory is increased by  \(Q_x\) \textbf{(if x is not in the inventory?)}. Similarly for outbound transactions, ID\_Product and its Quantity attribute value are extracted from the relative Order and the correspondent value is decreased in Inventory.
    \item \textbf{show and generate the current inventory balance or stock inquiries:} The entity Inventory has data related to the number of each product which are stored. An employee can access the instances of this entity and review the attribute Element\_quantity to do stock inquiries \textcolor{Red}{Review} 
    \item \textbf{receive and process the Customers order, specifying which products they want and respective quantity:} Salesmen who are responsible of making orders check if there is enough quantity of the product. If there is enough, they insert in Entity Order a new row with attributes mentioned in the Entities Table. The attribute Id\_order identifies each order.
    \item \textbf{modification and cancellation of orders:} the salesman can change or cancel the order by the Attribute ID\_Order of Entity Order.
    \item \textbf{allow users to view order and shipment status of finished products:} With the unique tracking number the attribute ID\_Shipment, the Costomers can get information about the order and shipment.
    \item \textbf{generate invoice whenever payment has been made:} When a customer pays an order, there is an insertion in  Entity Invoice. Each invoice is identified by the attribute Id\_invoice.
    \item \textbf{permit transfer of items and products:} Entity Customer has an attribute Address which refers to where the ordered products are going to be shipped. A customer pays an Invoice and each Invoice is associated to an Order.
    \item \textbf{grant Cycle Counting in order to validate the accuracy of inventory:} \textcolor{Red}{TO DO}
    \item \textbf{re-ordering the previous orders is allowed:} The system allows salesman to reorder orders. that means, the inventory system allows customers to save their orders the Attribute ID\_Order of Entity Order and access them again through the salesman and the Atribute Order\_date will be updated by new order's date.
    \item \textbf{create tracking code for orders:} Attribute ID\_Order of Entity Order store an unique identifier of each order. It is shown too in the relationship Ships between Order, Employee and Shipment.
\end{itemize}




