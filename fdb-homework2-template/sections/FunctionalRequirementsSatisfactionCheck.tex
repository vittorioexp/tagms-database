\subsection{Functional Requirements Satisfaction Check}

The DBMS has to be able to:
\begin{itemize}
	\item \textbf{store all the details of the employees, customers and suppliers in the organization:} Entities Employee and Role store data related to the employees. Entity Customer has details about the customers and entity Supplier has data related to the Supplier.
	\item \textbf{allow the employees to update their personal information:} Entity Employee has some attributes as Email\_address, Password or Phone\_number which can be changed. Employees can access the system using their credentials which are Email\_address and Password and change this data.
	\item \textbf{store details of all on-hand products in the warehouse such as item code, item description, quantity and expiration date:} Attributes ID\_Product, Description, Expiration\_Date, Product\_Cost from entity Product and ID\_Item, Description from entity Item show this data. The amount of each item is shown in attribute Item\_Quantity of entity Item. Assembled products are organized in lots whose information is contained in Lot entity as attributes ID\_Lot, ID\_Product (the product that lot is composed of), Quantity (the amount of elements of product the lot is composed of), Expiration\_Date (common for all elements of the lot) and Order (the order the lot is assigned to, if NULL determines a not-yet-assigned lot.  
	\item \textbf{allow the employees to log into the system and enter the inbound items they received with information item code, item description, quantity, expiration date and supplier:} With attributes Email\_address and Password employees log in the database and insert this data in the entity Item, that could be inserting a new instance or updating an existing one.
	\item \textbf{show and generate the list of inbound and outbound transactions:} the inbound transactions can be derived from instances of the Contract entity, the outbound transactions can be derived from instances of the Order entity.
	\item \textbf{allow the employees to log into the system and enter the outbound transaction needed for the issuance of the products in the production and shipment to the customers:} Salesmen are responsible of entering outbound transactions as instances of the Order entity with proper attributes values, specifically Product\_List, Quantities\_List (has at index \textit{i} the quantity of purchased items for product at index \textit{i} of Product\_List), Address and ID\_Order which identifies the instance, Cost and Tax\_Percentage. The system will calculate the not-yet-assigned lots that satisfy the requirements and collect their ID in the attribute Lots\_List of Order (and also update Order attribute in Lot entity). Cost is a derived attribute, compute automatically given the composition of the lot and the product cost.
    \item \textbf{inventory stocks will be automatically updated whenever there are inbound and outbound transactions:} the update is executed automatically when an inbound transaction happens by inspecting each ID\_Item \textbf{x} and associated Quantity \(Q_x\) in the relative Contract, then for each said \textbf{x} its Item\_Quantity attribute in the entity Item is increased by  \(Q_x\) \textbf{(if x is not in the inventory?)}. For outbound transactions, since each ID\_Product involved and its Quantity attribute value are extracted from the relative Order and the correspondent value is decreased in the relative Item instance.
    \item \textbf{show and generate the current inventory balance or stock inquiries:} The entity Item has data related to the quantity of each item stored. The entity Lot has data about the quantity of products each lot contains, and which lots are in the stock or not-shipped yet. This latter information can be retrieved checking if there exist a relationship "Ship" that involves the order a lot is associated to, since this relationship is created once the order is actually shipped to the customer. An employee can access the instances of these entity and specifically the attributes Item\_quantity of Item and Product\_Quantity of Lot to determine the exact current content of the stock.
    \item \textbf{receive and process the Customers order, specifying which products they want and respective   quantity:} Salesmen are able to access the database and enter an instance of the Order entity reporting the desired list of products and the respective quantities. The system will assign appropriate lots if they already exist, and if they do not exist the Salesman will wait and check periodically. As soon as they are ready, the order will be entered and will be uniquely determined by ID\_Order.
    \item \textbf{modification and cancellation of orders:} the salesman can change or cancel the order accessing it through the Attribute ID\_Order of Entity Order. The order will be canceled after a certain period of time after the customer does not pay for the order.
    \item \textbf{allow users to view order and shipment status of finished products:} With the unique tracking number the attribute ID\_Shipment of the Ship relationship, and the unique Id\_order attribute of the Order entity the users can get information about the order and shipment.
    \item \textbf{generate invoice whenever payment has been made:} When a Salesman creates an order, it also sets to False a boolean attribute "Payed" that determined a not-payed-yet order. When the customer pays the order, the information about the payment is inserted as attributes of the Place relationship. The invoice document can be generated instantly extracting the information from the related instance of the Order entity. The subtotal is computed given the attributes Cost and Tax\_Percentage.
    \item \textbf{permit transfer of items and products:} Entity Customer has an attribute Address which refers to where the ordered products are going to be shipped. When the customer pays the order and so the attribute Payed of the Order entity is set to True, the Order information is forwarded to the workers that access to the related Address, set up the shipment and create a Ship relationship with ID\_Shipment
    \item \textbf{grant Cycle Counting in order to validate the accuracy of inventory:} \textcolor{Red}{TO DO}
    \item \textbf{re-ordering the previous orders is allowed:} The system allows salesman to reorder orders. that means, the inventory system allows customers to save their orders the Attribute ID\_Order of Entity Order and access them again through the salesman, then the Attribute Order\_date will be updated by new order's date.
    \item \textbf{create tracking code for orders:} Attribute ID\_Order of Entity Order store an unique identifier of each order. It is shown too in the relationship Ships between Order, Employee and Shipment.
\end{itemize}





