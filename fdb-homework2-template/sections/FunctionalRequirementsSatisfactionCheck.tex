\subsection{Functional Requirements Satisfaction Check}

The DBMS has to be able to:
\begin{itemize}
	\item \textbf{store all the details of the employees, customers and suppliers in the organization:} Employee entity stores data related to the employees. Customer entity has details about the customers and Supplier entity has data related to suppliers.
	\item \textbf{allow the employees to update their personal information:} Employee entity has some attributes as Email\_address, Password or Phone\_number that can be changed. Employees can access the system using their credentials and change this data.
	\item \textbf{store details of all on-hand products in the inventory such as item code, item description, quantity and expiration date:} The inventory is represented by the entities Item, Product, Package and Lot. The Item entity contains information relating to materials (e.g., ingredients, packaging materials, ...) with their respective descriptions and quantities. Likewise, the Product entity contains information relating to finished products and the Package entity contains information relating to packaging. The products are packed in lots. Each lot is also characterized by an expiry date.
	\item \textbf{allow the employees to log into the system and enter the inbound items they received with information item code, item description, quantity, expiration date and supplier:} Employees can log in the database and insert data about new items in the system. An employee can also update an existing Item and its respective quantity.
	\item \textbf{show and generate the list of inbound and outbound transactions:} The inbound transactions can be generated by inspecting the instances of the Contract entity, while the outbound transactions can be obtained by inspecting the instances of the Order entity.
	\item \textbf{allow the employees to log into the system and enter the outbound transaction needed for the issuance of the products in the production and shipment to the customers; inventory stocks will be automatically updated whenever there are inbound and outbound transactions; show and generate the current inventory balance or stock inquiries: } Regarding items, the update is executed automatically when an inbound transaction occurs by inspecting the new Contract: for each Item the quantity "Item\_Quantity" is increased accordingly. Regarding the outbound transaction, the value of "Item\_Quantity" is decreased when a new lot (which stocks a product that is made up of that Item) is prepared. Regarding products, the stock quantity can be obtained by inspecting the lots produced but not yet ordered or expired. The same holds true for packages.

    \item \textbf{receive and process the Customers order, specifying which products they want and respective quantity:} Salesmen are able to access the database and enter an instance of the Order entity reporting the desired list of products and the respective quantities. The Salesman will find and assign appropriate lots if they already exist, and if they do not exist it will notify the Production department which will notify back when they are ready. As soon as this happens, the order will be entered and will be uniquely determined by ID.
    \item \textbf{modification and cancellation of orders:} the salesman can change or cancel the order accessing it through the Attribute ID of Order entity.
    \item \textbf{allow users to view order and shipment status of finished products:} With the unique tracking number the attribute "Tracking number" of the Ship relationship, and the unique ID attribute of the Order entity the users can get information about the order and shipment.
    \item \textbf{generate invoice whenever payment has been made:} When a Salesman creates an order, it sets automatically to False a boolean attribute Order\_paid that determines a not-payed-yet order. When the customer pays the order, the information about the payment is inserted as attributes of the Place relationship. The invoice document can be generated instantly extracting the information from the related instance of the Order entity. Final\_Price is computed given the attributes Net\_Price and Taxes.
    \item \textbf{permit transfer of items and products:} Entity Customer has an attribute Address which refers to where the ordered products are going to be shipped. When the customer pays the order and so the attribute Order\_paid of the Order entity is set to True, the Order information is forwarded to the workers that inspect the related Address, set up the shipment and create a Ship relationship with Tracking number
    \item \textbf{grant Cycle Counting in order to validate the accuracy of inventory:} Cycle counting is define as a periodic check by the Managers or by a Worker Supervisor on the items and products in the warehouse. After acquiring the actual quantities for each item and the list of lots in the stock, it will be reconciled with the system quantities. The Manager can update (if necessary) the values of such attributes in the database to the correct value.
    \item \textbf{re-ordering the previous orders is allowed:} The system allows salesmen to access past orders using the ID attribute and retrieve information about the lots, the products, their quantities and all the necessary data to set up a new order with the same content.
    \item \textbf{create tracking code for orders:} Attribute Tracking number of Ship relationship stores an identifier, provided by the third-party company responsible for the shipping, that uniquely identifies the shipment.
\end{itemize}



