\subsection{Functional Requirements Satisfaction Check}

The DBMS has to be able to:
\begin{itemize}
	\item \textbf{store all the details of the employees, customers and suppliers in the organization:} Employee and Role entities store data related to the employees. Customer entity has details about the customers and Supplier entity has data related to the Supplier.
	\item \textbf{allow the employees to update their personal information:} Employee entity has some attributes as Email\_address, Password or Phone\_number that can be changed. Employees can access the system using their credentials and change this data.
	\item \textbf{store details of all on-hand products in the inventory such as item code, item description, quantity and expiration date:} The concept of inventory is implemented by the entities Item, Product and Lot, whose attributes show the data regarding the stock. The amount of each item is shown in attribute Quantity of entity Item. Assembled products are organized in lots whose information is contained in Lot entity in the attributes ID, Product\_ID (the product that lot is composed of), Product\_quantity (the amount of elements of product the lot is composed of), Expiration\_date (common for all elements of the lot) and Order\_ID (the order the lot is assigned to, if NULL determines a not-yet-assigned lot).  
	\item \textbf{allow the employees to log into the system and enter the inbound items they received with information item code, item description, quantity, expiration date and supplier:} Employees can log in the database and insert this data in the entity Item, that could be inserting a new instance or updating an existing one.
	\item \textbf{show and generate the list of inbound and outbound transactions:} the inbound transactions can be derived from instances of the Contract entity, the outbound transactions can be derived from instances of the Order entity.
	\item \textbf{allow the employees to log into the system and enter the outbound transaction needed for the issuance of the products in the production and shipment to the customers:} Salesmen are responsible of entering outbound transactions, that are instances of the Order entity with proper values for the attributes. The salesman will check for not-yet-assigned lots that satisfy the requirements and initialize a relationship between lot and order by updating the Order attribute in Lot entity. Net\_price is a derived attribute, computed automatically given the composition of the lot and the product cost.
    \item \textbf{inventory stocks will be automatically updated whenever there are inbound and outbound transactions:} the update is executed automatically when an inbound transaction happens by inspecting each ID\_Item \textbf{x} and associated Quantity \(Q_x\) in the relative Contract, then for each said \textbf{x} its Item\_Quantity attribute in the entity Item is increased by  \(Q_x\). For outbound transactions, lots with NULL value in the order attribute are in the stock and unassigned, and lots associated to an order that is not in a Ship relationship are the ones that are waiting in the stock to be shipped. Therefore, these two sources of information allow to determine the lots in the stock at every time.
    \item \textbf{show and generate the current inventory balance or stock inquiries:} The Item entity has data related to the quantity of each item stored. The Lot entity has data about the quantity of products each lot contains, and which lots are in the stock or not-shipped yet. This latter information can be retrieved checking if there exist a relationship "Ship" that involves the order a lot is assigned to, since this relationship is created only once the order is actually shipped to the customer. An employee can access the instances of these entity and specifically the attributes Quantity of Item entity and Product\_quantity of Lot entity (eventually, of all lots containing a certain product) to determine the exact current content of the stock.
    \item \textbf{receive and process the Customers order, specifying which products they want and respective quantity:} Salesmen are able to access the database and enter an instance of the Order entity reporting the desired list of products and the respective quantities. The Salesman will find and assign appropriate lots if they already exist, and if they do not exist it will notify the Production department which will notify back when they are ready. As soon as this happens, the order will be entered and will be uniquely determined by ID.
    \item \textbf{modification and cancellation of orders:} the salesman can change or cancel the order accessing it through the Attribute ID of Order entity.
    \item \textbf{allow users to view order and shipment status of finished products:} With the unique tracking number the attribute "Tracking number" of the Ship relationship, and the unique ID attribute of the Order entity the users can get information about the order and shipment.
    \item \textbf{generate invoice whenever payment has been made:} When a Salesman creates an order, it sets automatically to False a boolean attribute Order\_paid that determines a not-payed-yet order. When the customer pays the order, the information about the payment is inserted as attributes of the Place relationship. The invoice document can be generated instantly extracting the information from the related instance of the Order entity. Final\_Price is computed given the attributes Net\_Price and Taxes.
    \item \textbf{permit transfer of items and products:} Entity Customer has an attribute Address which refers to where the ordered products are going to be shipped. When the customer pays the order and so the attribute Order\_paid of the Order entity is set to True, the Order information is forwarded to the workers that inspect the related Address, set up the shipment and create a Ship relationship with Tracking number
    \item \textbf{grant Cycle Counting in order to validate the accuracy of inventory:} Cycle counting is meant as a periodic check by the Managers or by a Worker Supervisor on the items and products in the warehouse. After collecting the real quantities for each item and the list of lots in the stock, the Manager can update (if necessary) the values of such attributes in the database to the correct value.
    \item \textbf{re-ordering the previous orders is allowed:} The system allows salesmen to access past orders using the ID attribute and retrieve information about the lots, the products, their quantities and all the necessary data to set up a new order with the same content.
    \item \textbf{create tracking code for orders:} Attribute Tracking number of Ship relationship stores an identifier, provided by the third-party company responsible for the shipping, that uniquely identifies the shipment.
\end{itemize}








