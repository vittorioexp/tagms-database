\subsection{Functional Requirements Satisfaction Check}

The DBMS has to be able to:
\begin{itemize}
	\item \textbf{store all the details of the employees, customers and suppliers in the organization:} Employee entity stores data related to the employees. Customer entity has details about the customers and Supplier entity has data related to suppliers.
	\item \textbf{allow the employees to update their personal information:} Employee entity has some attributes as Email\_address, Password or Phone\_number that can be changed. Employees can access the system using their credentials and change this data.
	\item \textbf{store details of all on-hand products in the inventory such as item code, item description, quantity and expiration date:} The inventory is represented by the entities Item, Product, Package and Lot. The Item entity contains information relating to materials (e.g., ingredients, packaging materials, ...) with their respective descriptions and quantities. Likewise, the Product entity contains information relating to finished products and the Package entity contains information relating to packaging. The products are packed in lots. Each lot is also characterized by an expiry date.
	\item \textbf{allow the employees to log into the system and enter the inbound items they received with information item code, item description, quantity, expiration date and supplier:} Employees can log in the database and insert data about new items in the system. An employee can also update an existing Item and its respective quantity.
	\item \textbf{show and generate the list of inbound and outbound transactions:} The inbound transactions can be generated by inspecting the instances of the Contract entity, while the outbound transactions can be obtained by inspecting the instances of the Order entity.
	\item \textbf{allow the employees to log into the system and enter the outbound transaction needed for the issuance of the products in the production and shipment to the customers; inventory stocks will be automatically updated whenever there are inbound and outbound transactions; show and generate the current inventory balance or stock inquiries: } Regarding items, the update is executed automatically when an inbound transaction occurs by inspecting the new Contract: for each Item the quantity "Item\_Quantity" is increased accordingly. Regarding the outbound transaction, the value of "Item\_Quantity" is decreased when a new lot (which stocks a product that is made up of that Item) is prepared. Regarding products, the stock quantity can be obtained by inspecting the lots produced but not yet ordered or expired. The same holds true for packages.
    \item \textbf{receive and process the Customers order, specifying which products they want and respective quantity:} Salesmen are able to access the database and enter an instance of the Order entity reporting the lots containing the desired products only when all lots are ready.
    \item \textbf{allow users to view order and shipment status of finished products; create tracking code for orders:} With the unique tracking number (attribute "Track\_num" of the Ship relationship), and the unique ID attribute of the Order entity, the users can get information about the order and shipment.
    \item \textbf{generate invoice whenever payment has been made:} When an order is placed, the invoice is automatically generated by the application connected to the system. The data is extracted from the entities Order and Lot, and from the "Draws from" relationship. The total amount, net price, taxes and the list of ordered Lots are specified.
    \item \textbf{grant Cycle Counting in order to validate the accuracy of inventory:} Cycle counting is a periodic check done by a warehouse worker on the items in the physical inventory. After acquiring the real quantities for each item, a check will be made on the system. In case of mismatch, the "Quantity" attribute of the Item is updated. 
    \item \textbf{re-ordering the previous orders is allowed:} The system allows salesmen to access past orders and lots using the ID attribute and retrieve information about the lots, the products, and their quantities. In this way, the customer can order the same goods. 
\end{itemize}



