\subsection{Functional Requirements Satisfaction Check}

The DBMS has to be able to:
\begin{itemize}
	\item \textbf{store all the details of the employees, customers and suppliers in the organization:} Entities Employee and Role store data related to the employees. Entity Customer has info about the customers and entity Supplier has data related to the Supplier.
	\item \textbf{allow the employees to update their personal information:} Entity Employee has some attributes as Email, Password or Phone Number which can be changed.
	\item \textbf{store details of all on-hand products in the warehouse such as item code, item description, quantity and expiration date:} Entities Product, Item and Inventory and the relationship Stocked store this data. 
	\item \textbf{allow the employees to log into the system and enter the inbound items they received with information item code, item description, quantity, expiration date and supplier:} With attributes Email and Password employees log in the application and insert this data in Entities Product,Item and Inventory.
	\item \textbf{show and generate the list of inbound and outbound transactions:}
	\item \textbf{allow the employees to log into the system and enter the outbound transaction needed for the issuance of the products in the production and shipment to the customers:}
    \item \textbf{inventory stocks will be automatically updated whenever there are inbound and outbound transactions:}
    \item \textbf{show and generate the current inventory balance or stock inquiries:}
    \item \textbf{receive and process the Customers order, specifying which products they want and respective quantity:}
    \item \textbf{modification and cancellation of orders:}
    \item \textbf{allow users to view order and shipment status of finished products:}
    \item \textbf{generate invoice whenever payment has been made}
    \item \textbf{permit transfer of items and products:}
    \item \textbf{grant Cycle Counting in order to validate the accuracy of inventory:}
    \item \textbf{re-ordering the previous orders is allowed:}
    \item \textbf{create tracking code for orders:}
\end{itemize}




The system must store
\begin{itemize}
	\item Customer data:
\end{itemize}
\begin{itemize}
	\item Employee data with its activity:
	\item Any action of the employee on the order will be stored on the Order entity.
\end{itemize}

The system must allow Customers to:
\begin{itemize}
	\item View orders and shipping specifications
	\item Customer will be able to check where the order is with the tracking number.
\end{itemize}

The system must allow Employee to:
\begin{itemize}
	\item Login to system using the email address and password
	\item Modify/delete the order
	\item modify their personal information
\end{itemize}