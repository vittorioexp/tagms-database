\begin{longtable}{|p{.20\columnwidth}|p{.20\columnwidth} |p{.30\columnwidth}|p{.25\columnwidth} |}
    \hline
    \textbf{Entity} & \textbf{Description} & \textbf{Attributes} & \textbf{Identifier}  \\\hline

    Employee & Represents data of an employee who works in the company and needs access to the system &
    \begin{itemize}
        \vspace{-1em}
        \item Badge\_number:   badge id of the employee, serial
        \item First\_name:   name of the employee, text
        \item Last\_name:   surname of the employee, text
        \item Phone\_number:   phone number of the employee, int
        \item Email\_address:   email address of the employee, text
        \item Gender:   gender of the employee, bit
        \item Birth\_date:   birth date of the employee, datetime
        \item Hiring\_date:   hiring date of the employee, text
        \item Role\_ID:   role identifier, int
        \item Department\_ID:   department identifier, int
    \end{itemize}
    &  Badge\_number \\\hline

    Role & Represents data on the role of employees who work in the company &
    \begin{itemize}
        \vspace{-1em}
        \item ID:   role ID of the employee, serial
        \item Name:   name o the role, text
        \item Description:   technical description of the role, text
    \end{itemize}
    &  ID \\\hline

    Department & Represents data on the departments in which employees work &
    \begin{itemize}
        \vspace{-1em}
        \item ID:   department ID of the company, serial
        \item Name:   name of the department, text
        \item Description:   description of the department's function, text
    \end{itemize}
    &  ID \\\hline

    Customer & Represents data about a customer of the company &
    \begin{itemize}
        \vspace{-1em}
        \item ID:   customer ID, serial       %TODO: should we use the email as primate key?
        \item Customer\_name:   name of the customer, text
        \item Phone\_number:   phone number of the customer, int
        \item Email\_address:   email address of the customer, text
        \item Address:   %TODO: dobbiamo collegarlo all'entità location?
        \item Registration\_date:   customer registration date in the database, datetime
    \end{itemize}
    &  ID \\\hline

    Contract & Represents data about a contract stipulated between a supplier and a manager for the supply of items &
    \begin{itemize}
        \vspace{-1em}
        \item ID:   contract ID, serial
        \item Name:   name of the contract, text            %TODO: what do you mean with name?
        \item Description:   description of the contract, text
        \item Contract\_date:    date of signature of the contract with the supplier, datetime
        \item Supplier\_ID:   supplier ID, serial
        \item Employee\_ID:   employee ID, serial
    \end{itemize}
    &  ID \\\hline

    Supplier & Represents data about a supplier of the company &
    \begin{itemize}
        \vspace{-1em}
        \item ID:   supplier ID, serial
        \item Supplier\_name:   name o the supplier company, text
        \item Phone\_number:   phone number of the supplier company, int
        \item Email\_address:   email address of the supplier company, text
        \item Address: %TODO: should we use location entity?
        \item Registration\_date:   recording date of the supplier company, datetime
    \end{itemize}
    &  ID \\\hline

    Order & Represents the order placed by a salesman for a customer &
    \begin{itemize}
        \vspace{-1em}
        \item ID:   order ID, serial
        \item Order\_date:   date in which the order has been processed, datetime
        \item Customer\_ID:   customer ID, serial
        \item Sub\_total:   total amount of the order, float            %tot = for each lot in the order ( production_price x qty of product x inflation / discount )
    \end{itemize}
    &  ID \\\hline

    %Payment & Represents the payment made by the customer &
    %\begin{itemize}
    %    \vspace{-1em}
    %    \item ID
    %    \item Payment\_method
    %    \item Payment\_status
    %    \item Invoice\_ID
    %    \item Order\_ID
    %\end{itemize}
    %&  ID \\\hline

    Lot & Represents the inventory of the company, containing products and items &
    \begin{itemize}
        \vspace{-1em}
        \item ID:   lot ID, serial
        \item Product\_ID:   product ID, serial
        \item Product\_quantity:   quantity of the produced product, int
        \item Package\_ID:   package ID related to rhe specific lot, serial
        \item Package\_quantity:   lot packaging quantity, int
        \item Expiration\_date:   expiration date of the specific product, datetime
    \end{itemize}
    &  ID \\\hline

    %Storage & Represents the physical storages that contain the elements present in the inventory &
    %    \begin{itemize}
    %        \vspace{-1em}
    %        \item ID
    %        \item Name
    %        \item Location
    %    \end{itemize}
    %&  ID \\\hline

    Product & Represents the final product that is marketed &
    \begin{itemize}
        \vspace{-1em}
        \item ID
        \item Name
        \item Description           %Nutritional values
        \item Production\_cost      %Cost of the production for a product
        \item Inflation             %Inflation i.e., price = inflation x production cost
        \item Category\_ID
    \end{itemize}
    &  ID \\\hline

    Item & Represents materials provided by suppliers from which the final products will be produced &
    \begin{itemize}
        \vspace{-1em}
        \item ID
        \item Description
        \item Category\_ID
    \end{itemize}
    &  ID \\\hline

    Package & Represents packaging of finished products which are made up of boxes, tapes, and other packaging materials &
    \begin{itemize}
        \vspace{-1em}
        \item ID
        \item Description
        \item Category\_ID %Do we shoould also count the quantity of the items by which the package is made of?
    \end{itemize}
    &  ID \\\hline

    Category & Represents the category of a product, an item or a package &
    \begin{itemize}
        \vspace{-1em}
        \item ID
        \item Name
        \item Description
    \end{itemize}
    &  ID \\\hline

    Invoice & Represents the invoice associated with the order of a customer &
    \begin{itemize}
        \vspace{-1em}
        \item ID
        \item Total\_amount     %TODO: same info is also inside ORDER
        \item Order\_ID
    \end{itemize}
    &  ID \\\hline

    Location & Represents the location where the customer resides, or where the company is located &
        \begin{itemize}
            \vspace{-1em}
            \item ID
            \item City
            \item Address
            \item Postal code    %TODO: add the foreign key of the other 3 entities: employee, customer and supplier
            \item Region
        \end{itemize}
    &  ID \\\hline
\end{longtable}