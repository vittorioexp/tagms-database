\section{Natural Language Sentences}
%what did we discover during the interviews? what characteristic should the system have?
%Notice that, the description here, although rich, is at high level.

A major international food and beverage company is investing in a new set of technologies to increase profits.
The company produces and distributes an average of X products every year, including drinks, juices, beers and snacks of various kinds.

The company's goal is to develop a new information management system that will help reduce excess and shortages of inventory. Furthermore, it is required to keep in memory as much information as possible about the commercial activity so that it is also possible to make analytics to monitor sales and expenses. In this system:
\begin{itemize}
    \item technical employees manage products in the inventory, invoices, and employees;
    \item sellers, on the other hand, have to deal with customers, orders, shipments, and payments;
    \item managers are in charge of managing suppliers and contracts;
    \item data analysts have access to the acquired data for inventory, cost and profit analysis;
    \item a system administrator will be responsible for managing the access privileges for each user on the system.
\end{itemize}

The figure of the Manager is important for the management of relationships (and therefore of contracts) with suppliers. The latter are intended to supply raw materials to the company. For example, if the company produces sugary drinks, suppliers will provide ingredients such as sugars, additives, artificial colors, water and natural flavors. In addition to these elements, there will be suppliers who will provide different types of packaging, such as glass, aluminum latins, or double corrugated cardboard boxes. Suppliers must be registered in the system, along with their name and contact details. Managers stipulate and maintain contracts with suppliers. These contracts are characterized by the following parameters: type of contract, duration of the contract, monetary agreement, quantity requested, and an identification for both the supplier and the raw material the first provides. It will therefore be possible for managers to inspect, for example through a list, all the contracts stipulated with the different suppliers. Furthermore, through the identification of the raw materials, managers have the possibility to filter contracts based on the type of product. Therefore, it is possible that the company can purchase the same type of product from different suppliers. Contracts stipulated between managers and suppliers are of different types:
\begin{itemize}
    \item Joint venture;
    \item Partnership;
    \item Collaboration-type network agreement;
    \item Network contract for joint operation.
\end{itemize}
Managers therefore work closely with suppliers and also with sellers. In fact, these two relationships allow the data analyst to optimize profits and limit losses. The data analyst has the task of carrying out some analyses on the internal data of the company to allow the latter to reduce the products that will be in excess (for example because they have expired) and to keep under control the minimum quantities for each single product in the warehouse. By periodically carrying out checks on the expiration date of the products in the database, the data analyst is able to find how many expired products are left in stock. The frequency on which this type of control is carried out will be in line with company policies. It is therefore possible that the company decides to carry out this check every quarter. In this case, at the end of the 3 months, the data analyst will carry out his analyzes, communicating the results to the managers through reports. In this way, in the event that the number of products discarded because expired is high, it will be the responsibility of the manager to reduce the quantities of raw materials purchased from the different suppliers. Furthermore, each product is characterized by a minimum quantity of stock in the warehouse. In the event that the number of products discarded is high, the manager will proceed with the decrease of this minimum quantity in stock, in such a way as to minimize losses. In the opposite case in which the number of orders is very high, it is possible that the quantities of some products are zeroed. In this scenario, the figure of the seller will be responsible for communicating this deficiency through a report to the managers. \textbf{The latter will modify the contract with the suppliers linked to the production of these products, increasing the quantity to be purchased.} Furthermore, the seller will be responsible for managing the relationship with the customer, proposing either to wait for new products to be made and then make a single order or to buy what remains in the warehouse and later make a second order. \textbf{However, it is important to underline that the stock inside the warehouse will be periodically checked by the data analyst, in order to make up for any quantity shortages of some products.} In both cases, therefore, the figure of the manager is of primary importance. In fact, it is he who, by decreasing or increasing (according to company policies) the quantities of raw materials purchased by suppliers, positively affects the total profit of the company. Furthermore, the data analyst, by inspecting the database, is able to provide order trends based, for example, on the status of the order. He will then be able to provide managers with sales trends within a certain period of interest, so that they can carry out market analyses.

The figure of the seller is important as well as it is the one who relates to the customer. In fact, he is in charge of interfacing with the customer \textbf{and receiving the different orders. Before placing the order in the system (database), the seller will have the task of checking the actual stock of the products concerned.} As explained above, he is in close contact with the manager to make up for any shortcomings. Assuming that the quantity of stock in question is sufficient, the seller will create a new order within the database. \textbf{Each order will be identified by a status}, with which it is monitored throughout the production and shipping cycle. Once the order is confirmed, the customer has a limited period (established following the company policies) to make the payment through an external payment system. \textbf{If the customer proceeds with the payment within the set period, the external system will notify the manager, the order will progress and the customer will receive a notification from the company containing the invoice.} In the event that the payment has not been made within this period, the order will be canceled. The company also offers the possibility to cancel the order only if the customer has not made the payment yet. If this happens, the order is marked as canceled. \textbf{If everything is successful, the manager takes care of creating (also in this case through an external shipping system) a new shipment, advancing the order status once again.} \textbf{By creating the shipment, the manager will receive a unique code, which will be communicated to the customer so that he can check in detail the status of the order.} In the event that the shipment is successful, the order will be considered successful. If, on the other hand, the package is lost or damaged, it will be the responsibility of the manager to manage the practice for the restoration to the company responsible for the transport, and then finally create a new order to be sent back to the customer.

\textbf{A customer should be able to register to a website in order to search for products to buy, view and manage its orders, and to write reviews -- made of a rating and some comments -- about products. Upon registering, it must specify its name, contact details, and delivery address. The customer receives an invoice with all the details when it places an order and can track it with a dedicated tracking number provided when the shipment is assigned to the courier. The factory shall keep track of all the payments made by customers, along with information about what payment method has been used. To ensure small orders to be evaded in an efficient manner, some products can be produced in advance, which means that the inventory should have a little stock of them, acting as a ``buffer''. If instead an order requires more products than how many are already in stock, the first is held pending until the requested goods are manufactured and packaged for the shipping.}

\textbf{In the inventory, items are stored with name, relative quantity, and expiration date. In particular, when there is a shortage of one or more items, the supplier providing it will be immediately notified with an automatic e-mail. Also, when a product is manufactured but has not been purchased yet, it should be stored in
the inventory as well, with name, quantity, and expiration date. When a shipment that contains a specific product leaves the factory, the latter must be deleted from the inventory. Of course, products -- as well as items -- that are expired, must be removed from the inventory as soon as possible and disposed of, as they cannot be sold anymore.}

%OLD TEXT:
%
%The company objective is to optimize costs by having only the bare minimum item quantities. In this way, there should be no waste of resources. The company produces and sells food and beverages. Each product has details, such as name and brand.\\
%
%The objective of the company is to implement an IMS to address overstocking and outages of items. To accomplish this, each item is stored in the system with name, relative quantity, and expiration date. In particular, when there is a shortage of one or more items, the supplier providing it will be immediately notified with an automatic e-mail. Also, when a product is manufactured but has not been purchased yet, it should be stored in the system as well, with name, quantity, and expiration date. When a shipment that contains a specific product leaves the factory, the latter must be deleted from the inventory.\\
%
%Some products can be produced in advance, which means that the inventory should have a little stock of them, acting as a ``buffer'', to be able to evade small orders smoothly. If an order requires more products than how many are already finished, the first is held pending until the requested goods are not manufactured and packaged for the shipping. Of course, products, as well as items, that are expired, must be removed from the inventory as soon as possible as they cannot be sold anymore.\\
%
%A customer should be able to register to a website in order to search for products to buy, view and manage its orders, and to write reviews -- made of a rating and some comments -- about products. Upon registering, it must specify its name, contact details, and delivery address. The customer receives an invoice with all the details when it places an order and can track it with a dedicated tracking number provided when the shipment is assigned to the courier. Suppliers must also be registered in the system, along with their name, contact details, and the agreement made with the factory stating which types of items the supplier will be responsible to provide.\\
%
%In the described system, the manufacturing and shipment of products is performed by simple workers, who are supervised by a manager. The latter is also responsible for finding and making agreements with suppliers for the replenishment of the inventory. Another important figure is the salesman, which is responsible for interacting with customers, advertising products and ensure that they are satisfied. In the end, there is the analyst, which is in charge of performing analyses and predictions on various Key Performance Indicators (KPIs), such as revenues, generated income for each product, and products rating, in order to determine which are the popular products and, on the contrary, which of them should be decommissioned. Each employee must be registered in the system with name, surname, contact details, the department in which he or she is working, and his or her role.